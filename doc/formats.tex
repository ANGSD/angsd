\documentclass[10pt]{article}
\usepackage{color}
\definecolor{gray}{rgb}{0.7,0.7,0.7}
\usepackage{framed}
\usepackage{enumitem}
\usepackage{longtable}
\usepackage[pdfborder={0 0 0},hyperfootnotes=false]{hyperref}

\addtolength{\textwidth}{3.4cm}
\addtolength{\hoffset}{-1.7cm}
\addtolength{\textheight}{4cm}
\addtolength{\voffset}{-2cm}


\begin{document}

\title{ANGSD formats}
\author{tsk}
\maketitle
\vspace*{1em}


\section{SAF formats}
SAF files are files that contain sample allele frequency. These are generated with -doSaf in main ANGSD. These contains either the loglikelihood ratio to the most likely category or the pp. This is determined if the -prior has been supplied.
The first 8 bytes magic number determines which SAF version. If no magic number is present then version0 is assumed.
\subsection{version 0}
First version of the SAF files were simply flat binary double files \texttt{PREFIX.saf} along with an associated \texttt{PREFIX.saf.pos.gz} which contains the gzip compressed 'chromosome position'. Assuming \emph{nChr} number of chromosomes, then we have \emph{nChr+1} categories for each site. The number of sites can therefore be deduced either directly from the number of lines in the uncompressed output of the \texttt{PREFIX.saf.pos.gz}, or by using the filesize (\emph{fsize}) of the \texttt{PREFIX.saf} $$\frac{fsize}{sizeof(double)*(nChr+1)}.$$
\subsection{version 1}
Second iteration of the saf files now contains two raw files and an index file. Still under development.
\begin{itemize}
\item[PREFIX.saf.gz] bgzf compressed flat floats. With similar interpretation as version0.
\item[PREFIX.saf.pos.gz] bgzf compressed flat integer. Representing the position.
\item[PREFIX.saf.idx] uncompressed binary file containing blocks of data described in \ref{tab1}.
\end{itemize}
\begin{table}
\begin{tabular}{rllll}
  \hline
  {\bf Col} & {\bf Field} & {\bf Type} & {\bf Brief description} \\
  \hline
  1 & {\sf CLEN} & size_t &  Length of CHR (not including terminating null)\\
  2 & {\sf CHR} & char* & Reference sequence name. Length is CLEN\\
  3 & {\sf NSITES} & size_t & Number of sites with coverage from reference CHR\\
  4 & {\sf OFF1} & long int & CHR offset into the PREFIX.saf.pos.gz \\
  5 & {\sf OFF2} & long int & CHR offset into the PREFIX.saf.gz \\
  \hline
\end{tabular}\label{tab1}
\caption{Content of entry for a single reference name in the PREFIX.saf.idx file.}
\end{table}
First 8 bytes in all three files is 8byte magic numer \emph{char[8]
  ``safv3''}. The next size_t value is the number of categories of the sample allelefrequency.

\end{document}
